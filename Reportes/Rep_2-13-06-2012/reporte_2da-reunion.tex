%% LyX 2.0.0 created this file.  For more info, see http://www.lyx.org/.
%% Do not edit unless you really know what you are doing.
\documentclass[english,spanish]{scrreprt}
\usepackage[T1]{fontenc}
\usepackage[latin9]{inputenc}
\usepackage[a4paper]{geometry}
\geometry{verbose,tmargin=10mm,bmargin=20mm,lmargin=10mm,rmargin=10mm,headheight=10mm,headsep=10mm,footskip=10mm}
\pagestyle{empty}
\setcounter{secnumdepth}{3}
\setcounter{tocdepth}{3}
\usepackage{amsmath}

\makeatletter

%%%%%%%%%%%%%%%%%%%%%%%%%%%%%% LyX specific LaTeX commands.
%% For printing a cirumflex inside a formula
\newcommand{\mathcircumflex}[0]{\mbox{\^{}}}


\makeatother

\usepackage{babel}
\addto\shorthandsspanish{\spanishdeactivate{~<>}}

\begin{document}

\title{Transcripci�n\\
Notas de la Tesis\\
2da Reuni�n}


\date{13/06/2012}

\maketitle

\chapter*{Introducci�n}

En esta reuni�n nos concentramos en modelar el conocimiento necesario
para representar la linea de pensamiento, que justifica las conclusiones
obtenidas del experimento \emph{La Frecuencia Al�lica en la Poblaci�n
no Influir�a en el Nro de Epitopes con Restricci�n para HLA-I Disponibles
en el virus HIV,} con las cuales se desarrollar�a un sistema que deber�a
verificar la validez de las conclusiones obtenidas.

Se investigo cual seria la informaci�n que deber�a otorgarsele a la
base de conocimiento como definiciones y reglas que el sistema tendr�a
para deducir. Se especificaron hipotesis, resultados y conclusiones. 


\chapter*{Borradores}


\section*{Conocimiento}


\subsection*{HLA }
\begin{itemize}
\item HLA identifica un organismo.
\item Clases de HLA caracterizan poblaciones geogr�ficas.
\item Existen 12 clases de HLA.
\item Cada poblaci�n geogr�fica se identifica por un subconjunto de los
\emph{12 HLA}.
\item Un HLA presenta p�ptidos en la superficie de la c�lula.
\item Cada clase de HAL presenta un subconjunto de p�ptidos.
\end{itemize}

\subsection*{HIV}
\begin{itemize}
\item Hay 10 tipos de HIV.
\item El HIV es un virus.
\item Los virus infectan c�lulas.
\item Una c�lula infectada produce prote�nas del virus que la infecta.
\end{itemize}

\subsection*{Geograf�a}
\begin{itemize}
\item Cada regi�n geogr�fica presenta tipos de HIV.
\end{itemize}

\subsection*{P�ptidos}
\begin{itemize}
\item Hay una funci�n $P$ que dado un HIV y un HLA retorna \# de p�ptidos
presentados por el HLA.
\item Una prote�na se descompone en secuencias de p�ptidos dentro de una
c�lula.
\end{itemize}

\subsection*{C�lula}
\begin{itemize}
\item Un p�ptido tiene asociado un lugar en la prote�na.
\end{itemize}

\section*{Hip�tesis}
\begin{itemize}
\item Se fijan dos regiones R1 y R2.

\begin{itemize}
\item Eso determina los $HLA^{Sets}(R_{i})$ y $HIV^{Sets}(R_{i})$.
\end{itemize}
\item Sean: 

\begin{itemize}
\item $hla\in HLA^{Sets}(R_{1})\backslash HLA^{Sets}(R_{2})$.
\item $hiv_{1}\in HIV^{Sets}(R_{1})\backslash HIV^{Sets}(R_{2})$. 
\item $hiv_{2}\in HIV^{Sets}(R_{1})\backslash HIV^{Sets}(R_{2})$.
\end{itemize}

Tenemos que:
\begin{itemize}
\item $P(hiv_{1},hla)\neq P(hiv_{2},hla)$.
\item $P(hiv_{2},hla)=8$.
\end{itemize}
\item Hip�tesis : $P(hiv_{1},hla)\neq P(hiv_{2},hla)\Longrightarrow(\triangle HLA\Longrightarrow\triangle HIV)$.
\end{itemize}

\section*{Resultado }


\section*{$\neg\triangle$}


\section*{Conclusi�n}

Los datos del experimento no validan la Hip�tesis, no son concluyentes.


\section*{Observaciones}
\begin{itemize}
\item $\triangle Geografia\overset{?}{\Longrightarrow}\triangle HLA$, $\triangle HLA\overset{?}{\Longrightarrow}\triangle HIV$.
\item $hla_{1}\Longrightarrow G_{1}\mathcircumflex hla_{1}\not\Longrightarrow G_{2}$.
\item $G_{1}$ hay \foreignlanguage{english}{$HIV_{1}$} y $G_{2}$ hay
$HIV_{2}$.
\item $\triangle HLA:\{HLA_{1},\emptyset\}$.
\item $P(HIV_{1},HLA_{1})$, $P(HIV_{2},HLA_{1})$.
\end{itemize}

\chapter*{Definiciones y Reglas}

\emph{TBD}
\end{document}
