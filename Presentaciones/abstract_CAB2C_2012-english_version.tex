%% LyX 2.0.0 created this file.  For more info, see http://www.lyx.org/.
%% Do not edit unless you really know what you are doing.
\documentclass[spanish,english]{paper}
\usepackage[T1]{fontenc}
\usepackage[latin9]{inputenc}
\usepackage{geometry}
\geometry{verbose,tmargin=1.75cm,bmargin=1.75cm,lmargin=1.75cm,rmargin=1.75cm,headheight=1.75cm,headsep=1.75cm,footskip=1cm}
\usepackage{babel}
\addto\shorthandsspanish{\spanishdeactivate{~<>}}

\usepackage[unicode=true]
 {hyperref}
\begin{document}

\title{FuL}

\maketitle
Alejandro Kondrasky$^{1}$, Daniel Gutson$^{1}$, Carlos E. Areces$^{1,2}$

$^{1}$FuDePAN: \foreignlanguage{spanish}{Fundaci�n para el Desarrollo
de la Programaci�n en �cidos Nucleicos}, X5002AOO, \foreignlanguage{spanish}{C�rdoba},
Argentina.

$^{2}$FaMAF, \foreignlanguage{spanish}{Universidad Nacional de C�rdoba,
Ciudad Universitaria}, X5000HUA, \foreignlanguage{spanish}{C�rdoba},
Argentina. 


\section*{Background}

The body of knowledge in biology, and of virology and immunology in
particular, is incremental in volume and complexity. This is why for
the production of that knowledge it will be useful to have it represented
in a formal language inside a knowledge base and utilize different
methodologies for analysis and manipulation, allowing verification
of the validity of the conclusions obtained by the results from experiments.

This is why we are developing FuDePAN's Logic processor (FuL) (http://ful.googlecode.com)
which objective is to organize, interpret, verify and explore the
knowledge in molecular biology applied to virology and immunology
in particular,so that we can found incongruence and automatically
derivate conclusions from that knowledge, being it's main function
the verification of conclusions obtained by results from experiments
using queries.

The test case will be the following conclusions obtained from experiments
done by FuDePAN:
\begin{itemize}
\item \textbf{\emph{Validate the conclusions obtained in the Junin experiment
about the temperature-change effects over the virus secondary structure:}}\foreignlanguage{spanish}{\emph{}}\\
Corroborate that the line of thought that include the predictions
of the effects of febrile state over the Junin RNA secondary structure,
in which is hypothesized that the temperature increment produce a
reduction in the production of nucleoproteins because the hairpin
loop in the intergenic region present dissimilar characteristics when
is compared the two ambisense genome strings when the temperature
is increased.
\end{itemize}
This tool has a model based on extensibility in plug-ins, allowing
replacement and adding of new functionalities to the system, simplifying
the adaptation of it to new types of knowledge. For that it will be
provided a API, which defines the way in which knowledge is flows
between the plug-ins and Ful's core, and a SDK composed of libraries
and tools required for building plug-ins.

The kernel of the tool will be composed of a planner based on PDDL
(Planning Domain Definition Language) semantic and a manager that
will be the intermediate between the plug-ins registered in the session
and the planner. Via a XML file, it can be possible to register the
plug-ins that FuL will utilize in that session and configure the session
variables, both from FuL and the plug-ins.

Also we will provide a knowledge representation language for virology
area base on DL(Description Logic), which will be used for representing
knowledge in the KB(Knowledge Base) to load in that session and make
queries to FuL. In particular, FuL will include a Semantic Reasoner
based on DL as a plug-in.


\section*{Referencias}
\begin{enumerate}
\item Franz Baader, Deborah L. McGuinness, Daniele Nardi, Peter F. Patel-Schneider:
\textbf{THE DESCRIPTION LOGIC HANDBOOK: Theory, implementation, and
applications}.
\item \textbf{The Seventh International Planning Competition Description
of Participant Planners of the Deterministic Track}, 2011. \href{http://www.plg.inf.uc3m.es/ipc2011-deterministic/ParticipatingPlanners}{www.plg.inf.uc3m.es/ipc2011-deterministic/ParticipatingPlanners}
\item Daniel Gutson, Agust�n March, Maximiliano Combina, Daniel Rabinovich:\textbf{
Prediction of consequences of the febrile status on the RNA secondary
structure of the Jun�n Virus}, 2006. \href{http://www.fudepan.org.ar/node/71}{www.fudepan.org.ar/node/71}\end{enumerate}

\end{document}
