\documentclass[portrait,final,a0paper,fontscale=0.277]{baposter}

\usepackage{calc}
\usepackage{graphicx}
\usepackage{amsmath}
\usepackage{amssymb}
\usepackage{relsize}
\usepackage{multirow}
\usepackage{rotating}
\usepackage{bm}
\usepackage{url}

\usepackage{graphicx}
\usepackage{multicol}
\usepackage{palatino}
\usepackage[utf8]{inputenc}
\newcommand{\captionfont}{\footnotesize}

\graphicspath{{images/}{../images/}}
\usetikzlibrary{calc}

\newcommand{\SET}[1]  {\ensuremath{\mathcal{#1}}}
\newcommand{\MAT}[1]  {\ensuremath{\boldsymbol{#1}}}
\newcommand{\VEC}[1]  {\ensuremath{\boldsymbol{#1}}}
\newcommand{\Video}{\SET{V}}
\newcommand{\video}{\VEC{f}}
\newcommand{\track}{x}
\newcommand{\Track}{\SET T}
\newcommand{\LMs}{\SET L}
\newcommand{\lm}{l}
\newcommand{\PosE}{\SET P}
\newcommand{\posE}{\VEC p}
\newcommand{\negE}{\VEC n}
\newcommand{\NegE}{\SET N}
\newcommand{\Occluded}{\SET O}
\newcommand{\occluded}{o}

%%%%%%%%%%%%%%%%%%%%%%%%%%%%%%%%%%%%%%%%%%%%%%%%%%%%%%%%%%%%%%%%%%%%%%%%%%%%%%%%
%%%% Some math symbols used in the text
%%%%%%%%%%%%%%%%%%%%%%%%%%%%%%%%%%%%%%%%%%%%%%%%%%%%%%%%%%%%%%%%%%%%%%%%%%%%%%%%

%%%%%%%%%%%%%%%%%%%%%%%%%%%%%%%%%%%%%%%%%%%%%%%%%%%%%%%%%%%%%%%%%%%%%%%%%%%%%%%%
% Save space in lists. Use this after the opening of the list
%%%%%%%%%%%%%%%%%%%%%%%%%%%%%%%%%%%%%%%%%%%%%%%%%%%%%%%%%%%%%%%%%%%%%%%%%%%%%%%%
\newcommand{\compresslist}{%
\setlength{\itemsep}{1pt}%
\setlength{\parskip}{1pt}%
\setlength{\parsep}{0pt}%
}

%%%%%%%%%%%%%%%%%%%%%%%%%%%%%%%%%%%%%%%%%%%%%%%%%%%%%%%%%%%%%%%%%%%%%%%%%%%%%%
%%% Begin of Document
%%%%%%%%%%%%%%%%%%%%%%%%%%%%%%%%%%%%%%%%%%%%%%%%%%%%%%%%%%%%%%%%%%%%%%%%%%%%%%

\begin{document}

%%%%%%%%%%%%%%%%%%%%%%%%%%%%%%%%%%%%%%%%%%%%%%%%%%%%%%%%%%%%%%%%%%%%%%%%%%%%%%
%%% Here starts the poster
%%%---------------------------------------------------------------------------
%%% Format it to your taste with the options
%%%%%%%%%%%%%%%%%%%%%%%%%%%%%%%%%%%%%%%%%%%%%%%%%%%%%%%%%%%%%%%%%%%%%%%%%%%%%%
% Define some colors
\definecolor{lightgreen}{RGB}{0,205,191}
\definecolor{lightgrey}{cmyk}{0,0,0,0.1}
\hyphenation{resolution occlusions}
%%
\begin{poster}%
  % Poster Options
  {
  % Show grid to help with alignment
  columns=2,
  grid=false,
  % Column spacing
  colspacing=1.25em,
  % Color style
  bgColorOne=lightgrey,
  bgColorTwo=lightgrey,
  borderColor=lightgreen,
  headerColorOne=black,
  headerColorTwo=lightgreen,
  headerFontColor=white,
  boxColorOne=white,
  boxColorTwo=lightgreen,
  % Format of textbox
  textborder=roundedleft,
  % Format of text header
  eyecatcher=true,
  headerborder=closed,
  headerheight=0.1\textheight,
%  textfont=\sc, An example of changing the text font
  headershape=roundedright,
  headershade=shadelr,
  headerfont=\Large\bf\textsc, %Sans Serif
  textfont={\setlength{\parindent}{1.5em}},
  boxshade=plain,
%  background=shade-tb,
  background=plain,
  linewidth=2pt
  }
  % Eye Catcher
  {
  	\includegraphics[height=6.5em]{ful_logo}
  }
  % Title
  {
  \LARGE
  \textsc{FuL: A Logic processor to aid design and validate virological experiments}\vspace{0.5em}
  }
  % Authors
  {
  	\textsc{Alejandro Kondrasky$^{1}$ - Daniel Gutson$^{1}$ - Carlos Areces$^{1,2}$}
    \footnotesize{
    \begin{flushleft}
	\hspace{1.5em}$^1$FuDePAN: Fundación para el Desarrollo de la Programación en Ácidos Nucleicos, X5002AOO, Córdoba, Argentina.\\
    \hspace{1.5em}$^2$FaMAF, Universidad Nacional de Córdoba, Ciudad Universitaria, X5000HUA, Córdoba, Argentina.\\
	\end{flushleft}
    }
  }		
  % University logo
  {% The makebox allows the title to flow into the logo, this is a hack because of the L shaped logo.
	\begin{minipage}{12em}
  	\includegraphics[width=\textwidth]{famaf}\\
  	\includegraphics[width=\textwidth]{fudepan_logo}
    \end{minipage}
  }

%%%%%%%%%%%%%%%%%%%%%%%%%%%%%%%%%%%%%%%%%%%%%%%%%%%%%%%%%%%%%%%%%%%%%%%%%%%%%%
  \headerbox{Introduction}{name=intro,span=2}{
%%%%%%%%%%%%%%%%%%%%%%%%%%%%%%%%%%%%%%%%%%%%%%%%%%%%%%%%%%%%%%%%%%%%%%%%%%%%%%
	The body of knowledge in biology, particularly in virology and
	immunology, is increasing in volume and complexity. This is why
	it would be useful to have these knowledge represented in a
	formal language inside a knowledge base. Subsequently, different
	methodologies for analysis and manipulation could be developed,
	allowing validity checks to be performed on conclusions obtained
	in experiments.
	
	FuDePAN's Logic processor(FuL) is
	being developed to organize, interpret, verify and explore
	knowledge in molecular biology, applied to virology and
	immunology in particular. This will help find inconsistencies and
	automatically derive new information. Its main function will be
	the verification of conclusions obtained by results from
	experiments using queries.
   }
   
%%%%%%%%%%%%%%%%%%%%%%%%%%%%%%%%%%%%%%%%%%%%%%%%%%%%%%%%%%%%%%%%%%%%%%%%%%%%%%
  \headerbox{1st Problem: Junín Virus}{name=problem1,below=intro,span=1}{
%%%%%%%%%%%%%%%%%%%%%%%%%%%%%%%%%%%%%%%%%%%%%%%%%%%%%%%%%%%%%%%%%%%%%%%%%%%%%%
	Our initial test case will be the following conclusion obtained
	from experiments done by FuDePAN, which will be the validation of the
	conclusions obtained in the Junín experiment about
	the temperature-change effects over the virus secondary
	structure:
	\\
    \includegraphics[width=\linewidth]{junin_problem}
    \small{
	Corroborate that the line of thought that includes the
	predictions of the effects of febrile state over the Junín RNA
	secondary structure, in which it is hypothesized that the
	temperature increment reduces the production of nucleoproteins
	because the hairpin loop in the intergenic region presents
	dissimilar characteristics when it is compared on the two
	ambisense genome strings when the temperature is increased.
	}
   }
%%%%%%%%%%%%%%%%%%%%%%%%%%%%%%%%%%%%%%%%%%%%%%%%%%%%%%%%%%%%%%%%%%%%%%%%%%%%%%
  \headerbox{1st Approach:}{name=solution1,below=intro,column=1}{
%%%%%%%%%%%%%%%%%%%%%%%%%%%%%%%%%%%%%%%%%%%%%%%%%%%%%%%%%%%%%%%%%%%%%%%%%%%%%%
   }
   
%%%%%%%%%%%%%%%%%%%%%%%%%%%%%%%%%%%%%%%%%%%%%%%%%%%%%%%%%%%%%%%%%%%%%%%%%%%%%%
  \headerbox{General Solution Approach}{name=generalsolution,column=0,below=problem1}{
%%%%%%%%%%%%%%%%%%%%%%%%%%%%%%%%%%%%%%%%%%%%%%%%%%%%%%%%%%%%%%%%%%%%%%%%%%%%%%
	\small{
	FuL has a plug-in architecture, simplifying the inclusion of new
	kinds of reasoning services. An API will be provided, which
	defines the way in which knowledge flows between the plug-ins and
	FuL's core reasoning engine. An SDK composed of libraries and
	tools required for building plug-ins will also be made available.
	
	The kernel of the tool will be composed of a planner that can
	handle PDDL (Planning Domain Definition Language) input, and a
	knowledge manager that will be the interface between the plug-ins
	registered in that session and the planner. Via a XML file, it
	will be possible to register the plug-ins that FuL will utilize
	in that session and configure different session parameters.
	
	FuL will include a semantic reasoner for Description Logics (DL)
	as one of the plug-ins, and we will also provide a knowledge
	representation language for the virology domain based on DL. This
	language will allow the development of an ontology of virology
	knowledge that will be available for querying during a FuL
	session.
	}
  }
%%%%%%%%%%%%%%%%%%%%%%%%%%%%%%%%%%%%%%%%%%%%%%%%%%%%%%%%%%%%%%%%%%%%%%%%%%%%%%
  \headerbox{Layers/Libraries Responsabilities}{name=layers,column=1,below=solution1}{
%%%%%%%%%%%%%%%%%%%%%%%%%%%%%%%%%%%%%%%%%%%%%%%%%%%%%%%%%%%%%%%%%%%%%%%%%%%%%%
  }
%%%%%%%%%%%%%%%%%%%%%%%%%%%%%%%%%%%%%%%%%%%%%%%%%%%%%%%%%%%%%%%%%%%%%%%%%%%%%%
  \headerbox{Materials and Methods}{name=method,column=1,below=layers}{
%%%%%%%%%%%%%%%%%%%%%%%%%%%%%%%%%%%%%%%%%%%%%%%%%%%%%%%%%%%%%%%%%%%%%%%%%%%%%%
  }
%%%%%%%%%%%%%%%%%%%%%%%%%%%%%%%%%%%%%%%%%%%%%%%%%%%%%%%%%%%%%%%%%%%%%%%%%%%%%%
  \headerbox{Repository}{name=source,column=0,below=generalsolution,above=bottom}{
%%%%%%%%%%%%%%%%%%%%%%%%%%%%%%%%%%%%%%%%%%%%%%%%%%%%%%%%%%%%%%%%%%%%%%%%%%%%%%
  \noindent
  \begin{minipage}{\linewidth}
  \begin{minipage}{0.73\linewidth}
    \indent{}The repository with the development documentation and soon source code is available at\\
  	\url{code.google.com/p/ful/}
  \end{minipage}\hfill%
  \begin{minipage}{0.25\linewidth}
  \hfill\includegraphics[width=\linewidth]{chart}
  \end{minipage}
  \end{minipage}
  }
%%%%%%%%%%%%%%%%%%%%%%%%%%%%%%%%%%%%%%%%%%%%%%%%%%%%%%%%%%%%%%%%%%%%%%%%%%%%%%
  \headerbox{References}{name=source,column=1,below=generalsolution,above=bottom}{
%%%%%%%%%%%%%%%%%%%%%%%%%%%%%%%%%%%%%%%%%%%%%%%%%%%%%%%%%%%%%%%%%%%%%%%%%%%%%%
    \footnotesize
    \bibliographystyle{ieee}
    \renewcommand{\section}[2]{\vskip 0.05em}
      \begin{thebibliography}{1}\itemsep=-0.05em
      \setlength{\baselineskip}{0.8em}
      \bibitem{franz:DL}
        \begin{minipage}{\linewidth}
        Franz Baader, Deborah L. McGuinness, Daniele Nardi, Peter F. Patel-Schneider.
  		\end{minipage}
        \begin{minipage}{\linewidth}
        The Description Logic Handbook: Theory, implementation, and applications.
  		\end{minipage}
      \bibitem{PDDL}
        \begin{minipage}{\linewidth}
        The Seventh International Planning Competition Description of Participant Planners of the Deterministic Track.
 		In {\em 2011}
  		\end{minipage}
  		\begin{minipage}{\linewidth}
		\begin{minipage}{0.87\linewidth}
		\bibitem{Junin}
      	  Daniel Gutson, Agustín March, Maximiliano Combina,\\ Daniel Rabinovich.\\
		  Prediction of consequences of the febrile status on the RNA secondary structure of the Junín Virus
           In {\em 2006} \url{www.fudepan.org.ar/node/71}
		\end{minipage}\hfill%
		\begin{minipage}{0.12\linewidth}
		\begin{center}
			\hfill\includegraphics[width=\linewidth]{junin_qrcode}
		\end{center}
		\end{minipage}
		\end{minipage}
      \end{thebibliography}
  }

\end{poster}

\end{document}

