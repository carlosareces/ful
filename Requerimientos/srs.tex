%% LyX 2.0.0rc3 created this file.  For more info, see http://www.lyx.org/.
%% Do not edit unless you really know what you are doing.
\documentclass[10pt,a4paper,spanish]{article}
\usepackage[T1]{fontenc}
\usepackage[latin9]{inputenc}
\pagestyle{headings}
\usepackage{amsmath}
\usepackage{amssymb}
\usepackage{nomencl}
% the following is useful when we have the old nomencl.sty package
\providecommand{\printnomenclature}{\printglossary}
\providecommand{\makenomenclature}{\makeglossary}
\makenomenclature

\makeatletter

%%%%%%%%%%%%%%%%%%%%%%%%%%%%%% LyX specific LaTeX commands.
\pdfpageheight\paperheight
\pdfpagewidth\paperwidth


%%%%%%%%%%%%%%%%%%%%%%%%%%%%%% User specified LaTeX commands.

\usepackage[toc, page, header]{appendix}\renewcommand{\appendixtocname}{Apendices}
\renewcommand{\appendixpagename}{Apendices}
\usepackage{amsfonts}

\makeatother

\usepackage{babel}
\addto\shorthandsspanish{\spanishdeactivate{~<>}}

\begin{document}

\title{FuL\\
Especificaci�n de Requerimientos de Software}


\author{Alejandro Kondrasky}

\maketitle
\pagebreak{}

\tableofcontents{}

\pagebreak{}


\section{Introducci�n}


\subsection{Prop�sito}


\paragraph{}

El prop�sito de este documento es la especificaci�n de requerimientos
de software en el marco de la tesis de grado de la carrera Lic. en
Cs. de la Computaci�n de FaMAF - UNC denominada \textbf{\emph{Procesador
L�gico para dise�o de experimentos de Virolog�a}}. 

Los requerimientos son provistos por integrantes de FuDePAN en su
car�cter de autores intelectuales de la soluci�n a implementar y colaboradores
de dicha tesis.


\subsection{Convenciones del Documento}

Las palabras clave \emph{DEBE}, \emph{NO DEBE}, REQUERIDO, DEBER�,
NO DEBER�, DEBER�A, NO DEBER�A, RECOMENDADO, PUEDE Y OPCIONAL en este
documento son interpretadas como esta descripto en el documento \emph{RFC
2119}.


\subsection{Audiencia Esperada}

A continuaci�n se enumeran las personas involucradas en el desarrollo
de la tesis, los cuales representan la principal audiencia de este
documento :
\begin{itemize}
\item Dr. Carlos Areces: Director de tesis, FaMAF 
\item Daniel Gutson: Colaborador de tesis, FuDePAN 
\item Alejandro Kondrasky: Tesista, FaMAF 
\end{itemize}

\subsection{Alcance del Producto}

El producto especificado en este documento se denomina \emph{Ful}\textbf{\nomenclature[FuL]{$FuL$}{FuDePAN Logic Processor}
}y su principal objetivo es, dada una \emph{KB\nomenclature[KB]{$KB$}{Knowledge Base (Base de Conocimiento)}}
determinada, organizar, analizar, chequear incongruencias en ella
y ser capas de con este planificar \emph{experimentos} \nomenclature[Experimento]{$Experimento$}{Pregunta (Mejorar luego)}.
A la vez que introducimos nuevos conocimientos a la KB, este deber�
pasar por los procesos previamente nombrados . 

El producto final debe proveer al usuario la capacidad de agregar
extensiones capaces de interpretar el conocimiento en la KB de maneras
apropiadas y de introducir nuevo conocimiento mediante un lenguaje
formal definido para ello. 

La principal responsabilidad de FuL es procesar el conocimiento presente
en la KB, utilizando las extensiones para interpretarlo, para as�
obtener una planificaci�n del experimento solicitado. Por otro lado,
las extensiones son responsables de interpretar ciertos tipos de conocimiento
en la KB y decidir si estos son consistentes o no.

En su versi�n inicial, FuL incluye una extensi�n de \emph{planning}\nomenclature[planning]{$planning$}{COMPLETAR}
y otra de \emph{DL}\nomenclature[DL]{$DL$}{Description Logics ( Logicas Descriptivas)}.


\subsection{Estructura del Documento}

La estructura de este documento sigue las recomendaciones de \emph{Gu�a
para la especificaci�n de requerimientos de la IEEE (IEEE Std 830-1998).
}Contiene las siguientes secciones :
\begin{itemize}
\item \emph{Secci�n 2: }Provee una descripci�n general de los aspectos generales
del producto, como la perspectiva de este, caracter�sticas principales.
\item \emph{Secci�n 3: }Describe las interfases del producto, tanto las
del usuario como interfases de software. 
\item \emph{Secci�n 4:} Organiza y describe los requerimientos funcionales
del producto.
\item \emph{Secci�n 5: }Organiza y describe los requerimientos no funcionales.
\end{itemize}

\section{Descripci�n General}


\subsection{Perspectiva del Producto}

Este producto trata de proveer a la comunidad cient�fica una herramienta
para organizaci�n, interpretaci�n, verificaci�n y exploraci�n del
conocimiento del �rea de virolog�a que se le otorga , para as� poder
encontrar incongruencias, conjuntos m�nimos que representen el mismo
conocimiento y conclusiones derivadas autom�ticamente de dicho conocimiento.

Otra de sus funciones principales es el planificar experimentos equivalentes
a preguntas del tipo $A\Longrightarrow B$.


\subsection{Caracter�sticas del Producto}


\subsubsection{Procesamiento continuo de la KB}

Provee la capacidad de continuar procesando la informaci�n presente
en la KB para as� obtener nuevas conclusiones del conocimiento disponible
en ella. Este proceso puede ser detenido y guardado para continuarlo
posteriormente.


\subsubsection{Minimizaci�n y optimizaci�n de la KB}

Permite reducir el conocimiento repetido y reinterpretar este de manera
tal de lograr obtener una nueva KB equivalente minimal.


\subsubsection{Introducir nuevo conocimiento a la KB}

Permite, mediante un lenguaje formal, introducir conocimiento a la
KB el cual sera chequeado para encontrar incongruencias con esta para
luego ser aceptado o mostrar la incongruencia encontrada.


\subsubsection{Planificar experimentos}

Provee la capacidad de planificar un experimento, encontrar las conclusiones
a los experimentos intermedios y proponer los experimentos intermedios
que se deben realizar para obtener una conclusi�n para el que fue
consultado inicialmente.


\subsubsection{API para la creaci�n de Plugins}

Provee una API para la creaci�n de plugins, los cuales ser�n utilizados
para interpretar el conocimiento que le es otorgado a dicha base de
datos. Esta API permite construir tanto el plugin que interpretara
el conocimiento como expandir el lenguaje formal para as� poder representar
nuevo tipo de conocimiento.
\end{document}
